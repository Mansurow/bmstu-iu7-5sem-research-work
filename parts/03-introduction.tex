\chapter*{Реферат}

\textbf{TODO: Написать реферат}

\chapter*{Термины}
\addcontentsline{toc}{chapter}{Термины}
Гетерогенная сеть - информационная сеть, в которой работают различные протоколы, используются технологии и оборудование различных фирм-производителей.

Интерфейс - формализованные правила, в соответствии с которыми взаимодействуют модули, реализующие протоколы соседних уровней модели сетевого взаимодействия (набор сервисов, предоставляемых данным уровнем соседнему).

Протокол - формализованные правила, определяющие последовательность и формат сообщений, которыми обмениваются сетевые компоненты, лежащие на одном уровне модели сетевого взаимодействия в разных узлах.

IP (Internet Protocol) — маршрутизируемый протокол сетевого уровня без установления соединения.

Сокет — пара IP-адрес: порт, однозначно определяющая сетевой процесс; точка доступа прикладного процесса к сети.

\chapter*{Обозначения и сокращения}
\addcontentsline{toc}{chapter}{Обозначения и сокращения}

В текущей расчетно-пояснительной записке применяется следующие сокращения и обозначения.

ОС --- Операционная система

IP ---  Internet Protocol 

OSI --- Open Systems Interconnection 

TCP --- Transmission Control Protocol

IETF --- Internet Engineering Task Force

BSD означает "Berkeley Software Distribution"

DNS
Аббревиатуры - ОС, TCP/IP, LAN, UUCP, IPX
Определения - фрагментации

\chapter*{Введение}
\addcontentsline{toc}{chapter}{Введение}

Организация взаимодействия между устройствами и программами в сети является сложной задачей.
Сеть объединяет разное оборудование, различные операционные системы и программы – это было бы невозможно без принятия общепринятых правил, стандартов.
В области компьютерных сетей существует множество международных и промышленных стандартов, среди которых следует особенно выделить международный стандарт OSI и набор стандартов IETF.

В ОС такую задачу реализуют сетевая подсистема, что позволяет иметь широкий спектр сетевых возможностей. Сетевая подсистема, выполняющаяся в режиме ядра, естественным образом ответственна за управление сетевыми устройствами ввода-вывода, но кроме этого на нее также возложены задачи маршрутизации и транспортировки пересылаемых данных.
Современные ОС Linux требует контроля по причине того, что безопасность ядра не идеальна в том числе и сетевая подсистема ядра \cite{moduls_kernel_bugs, version_kernel_bugs}. 
 
Обычно для отладки сетевых ошибок необходимо проверить все узлы, участвующие в сетевом взаимодействии: отправляющий, связующие и принимающий. Однако из-за сложной конфигурации, в сети возникают неочевидные связи и взаимодействий между различными элементами, что сильно осложняет процесс поиска источника неполадки даже в рамках одного узла. Неизбежно возникнет ситуация, в которой непонятно, с какой стороны подступиться проблеме. Тогда разработчику придется перебирать возможные причины возникновения ошибки, используя весь набор доступных для сетевой отладки инструментов и полагаясь лишь на свой профессиональный опыт или интуицию. Такой бессистемный подход приводит к высокой стоимости устранения сбоев.

\textbf{Целью работы} является провести анализ существующих средств мониторинга сетевых подсистем ядра ОС Linux.

Для достижения поставленной цели необходимо решить следующие задачи:
\begin{itemize}[label=---]
	\item провести анализ предметной области подсистем ядра ОС Linux;
	\item провести обзор существующих подсистем и средств сетевого мониторинга ядра OC Linux;
	\item сформулировать критерии сравнения средств сетевого мониторинга ядра;
	\item классифицировать существующие подсистемы и средства сетевого мониторинга.
\end{itemize}