\chapter*{Определения}
\addcontentsline{toc}{chapter}{Определения}
Гетерогенная сеть - информационная сеть, в которой работают различные протоколы, используются технологии и оборудование различных фирм-производителей.

Интерфейс - формализованные правила, в соответствии с которыми взаимодействуют модули, реализующие протоколы соседних уровней модели сетевого взаимодействия (набор сервисов, предоставляемых данным уровнем соседнему).

Протокол - формализованные правила, определяющие последовательность и формат сообщений, которыми обмениваются сетевые компоненты, лежащие на одном уровне модели сетевого взаимодействия в разных узлах.

IP (Internet Protocol) — маршрутизируемый протокол сетевого уровня без установления соединения.

Сокет — пара IP-адрес: порт, однозначно определяющая сетевой процесс; точка доступа прикладного процесса к сети.