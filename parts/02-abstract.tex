\section*{\centering РЕФЕРАТ}
\addcontentsline{toc}{section}{РЕФЕРАТ}
\setcounter{page}{2}

Научно-исследовательская работа \pageref{LastPage} с., \totalfigures\ рис., \totaltables\ табл., 17 ист., 1 прил.

ЯДРО LINUX, СЕТЕВАЯ ПОДСИСТЕМА, СЕТЕВОЙ МОНИТОРИНГ, МОДИФИКАЦИЯ КОДА ЯДРА, ЗОНДИРОВАНИЕ ЯДРА, ТОЧКИ ТРАССИРОВКИ, EBPF, СЕТЕВОЙ СТЕК, FTRACE, KPROBE 

Объект исследования — сетевая подсистема ядра Linux.

Цель работы --- Классификация методов сетевого мониторинга ядра Linux, выбор методов, которые наилучшим способом решают необходимые задачи.

Поставленная цель достигается путем рассмотрения и классификации существующих и применяемых методов сетевого мониторинга ядра Linux.