\section*{\centering РЕФЕРАТ}
\addcontentsline{toc}{section}{РЕФЕРАТ}
\setcounter{page}{2}

Научно-исследовательская работа \pageref{LastPage} с., \totalfigures\ рис., \totaltables\ табл., 17 ист., 1 прил.

ЯДРО LINUX, СЕТЕВАЯ ПОДСИСТЕМА, СЕТЕВОЙ МОНИТОРИНГ, МОДИФИКАЦИЯ КОДА ЯДРА, ЗОНДИРОВАНИЕ ЯДРА, ТОЧКИ ТРАССИРОВКИ, EBPF, СЕТЕВОЙ СТЕК, FTRACE, KPROBE.

Объект исследования --- сетевая подсистема ядра Linux.

Целью работы является провести анализ существующих средств мониторинга сетевой подсистемы ядра ОС Linux.
Поставленная цель достигается путем рассмотрения и классификации существующих и применяемых методов сетевого мониторинга ядра Linux.

В данной работе проводились изучение принципов работы сетевой подсистемы, средств и подсистем сетевого мониторинга ядра Linux, а также сравнение и анализ методов сетевого мониторинга, рассматриваемых на примере мониторинга пути сетевого пакета в сетевой подсистеме ядра Linux.  

Результат данной работы показал, что каждый метод сетевого мониторинга имеет ряд преимуществ и недостаткам и применим в зависимости от требований и ограничений к задаче.