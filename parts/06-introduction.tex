%\specsection{ВВЕДЕНИЕ}
\section*{\centering ВВЕДЕНИЕ}
\addcontentsline{toc}{section}{ВВЕДЕНИЕ}

Организация взаимодействия между устройствами и программами в сети сложная задача.
Сеть объединяет разные оборудование, операционные системы и программы --- это было бы невозможно без принятия общепринятых правил, стандартов.
В области компьютерных сетей существует множество международных и промышленных стандартов, среди которых следует особенно выделить международный стандарт OSI и набор стандартов IETF.

В ОС такую задачу реализуют сетевая подсистема, что позволяет иметь широкий спектр сетевых возможностей. Сетевая подсистема, выполняющаяся в режиме ядра, отвечает за управление сетевыми устройствами ввода-вывода, но кроме этого на нее также возложены задачи маршрутизации и транспортировки пересылаемых данных.
Современные ОС Linux требует контроля по причине того, что безопасность ядра не идеальна в том числе и сетевая подсистема ядра~\cite{version_kernel_bugs}. 
 
Для отладки сетевых ошибок проверяются все узлы, участвующие в сетевом взаимодействии: отправляющий, связующие и принимающий. Однако из-за сложной конфигурации, в сети возникают не очевидные связи и взаимодействий между различными элементами, что сильно осложняет процесс поиска источника неполадки даже в одном узле. Даже возникает ситуация, в которой непонятно, с какой стороны подступиться проблеме. Тогда разработчику придется перебирать возможные причины возникновения ошибки, используя набор доступных для сетевой отладки инструментов и полагаясь на профессиональный опыт разработчика. Такой бессистемный подход приводит к трудностям устранения сбоев .

Целью работы является провести анализ существующих средств мониторинга сетевой подсистемы ядра ОС Linux.

Для достижения поставленной цели необходимо решить следующие задачи:
\begin{itemize}[label=---]
	\item провести анализ предметной области сетевой подсистемы ядра ОС Linux;
	\item провести обзор существующих подсистем и средств сетевого мониторинга ядра OC Linux;
	\item сформулировать критерии сравнения средств сетевого мониторинга ядра;
	\item классифицировать существующие подсистемы и средства сетевого мониторинга.
\end{itemize}