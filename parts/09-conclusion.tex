%\specsection{ЗАКЛЮЧЕНИЕ}
\section*{\centering ЗАКЛЮЧЕНИЕ}
\addcontentsline{toc}{section}{ЗАКЛЮЧЕНИЕ}

В ходе данной работы были изучены:
\begin{itemize}
	\item структура и принципы работы сетевой подсистемы ядра Linux;
	\item методы сетевого мониторинга ядра Linux или средства и подсистемы мониторинга сетевой подсистемы ядра Linux;
	\item критерии сравнения методов сетевого мониторинга;
	\item принципы работы методов касательно сетевой подсистемы;
	\item преимущества и недостатки каждого из методов. 
\end{itemize}

Был выполнен обзор ядра Linux, его составных частей и сетевой подсистемы. Также проведен обзор и анализ существующих методов решений по сетевому мониторингу. Были сформирована критерии классификации методов сетевого мониторинга ядра Linux. Была проведена классификация методов сетевого мониторинга ядра Linux по критериям, сформированным в ходе работы.


